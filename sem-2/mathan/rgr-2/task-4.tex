\documentclass[a4paper,10pt]{article}
\usepackage[utf8]{inputenc}
\usepackage[english, russian]{babel}
\usepackage[T2A]{fontenc}
\usepackage{mathtools}

%opening
\title{РГР 2 -- Задание 4}
\author{Сандов Кирилл}
\date{Вариант 29}

\begin{document}

\maketitle

\section*{Задание 1}
Вычислить приближённо значение функции с точностью 0.0001:
$$
\sqrt[5]{36}
$$
\textbf{Решение:}
Рассмотрим функцию: $f(x)=(1 + x)^\alpha$. Известно её разложение в ряд Тейлора:
$$
f(x)=\sum_{n=0}^{\infty}{{\alpha \choose n}x^n}=1+\alpha x + \frac{\alpha(\alpha-1)}{2!}x^2 + ...\,+\frac{\alpha(\alpha-1)...(\alpha-n+1)}{n!}x^n +\,...
$$
Данное разложение верно, если ряд сходится, то есть при $|x|\leq1$.
Функция $f$ будет представлять собой исходное выражение из условия задания при $\alpha = \frac{1}{5}$, $x = 35$. Однако $x=35$ не входит в интервал сходимости ряда Тейлора $f$. Преобразуем наше выражение:
$$
\sqrt[5]{36} = \sqrt[5]{32(1+\frac{4}{32})} = 2\sqrt[5]{1+\frac{1}{8}}
$$
Теперь $x = \frac{4}{32} < 1$ и разложение в ряд Тейлора верно. Вычисляем:
$$
\sqrt[5]{36} = 2(1+\frac{1}{8})^\frac{1}{5} = 2(1 + \frac{1}{5\cdot8} + \frac{\frac{1}{5}\cdot(\frac{1}{5} - 1)}{2!}\cdot(\frac{1}{8})^2 + \frac{\frac{1}{5}\cdot(\frac{1}{5} - 1)\cdot(\frac{1}{5} - 2)}{3!}\cdot(\frac{1}{8})^3 +
$$
$$
+ \frac{\frac{1}{5}\cdot(\frac{1}{5} - 1)\cdot(\frac{1}{5} - 2)\cdot(\frac{1}{5} - 3)}{4!}\cdot(\frac{1}{8})^4) \approx 2(1 + 0.025 - 0.00125 + 0.00009375 - 0.000008) \approx
$$
$$
\approx 2 + 0.05 - 0.0025 + 0.00018 - 0.00001 \approx 2.0476
$$
Нам хватило сложения пяти членов ряда для достижения заданной точности, так как пятый член был уже меньше 0.0001, причём даже умноженный на два. Все дальнейшие члены были бы гарантированно меньше пятого члена, что не изменило бы число до заданной точности.
\\
\textbf{Ответ:} 2.0476.

\section*{Задание 2}
Разлагая подынтегральную функцию в степенной ряд, вычислить приближённо интеграл с точностью 0.0001:
$$
\int\limits_{0}^{\frac{1}{2}}\frac{\sin x}{x}dx
$$
\textbf{Решение:}
Известно разложение в ряд Тейлора функции $\sin x$:
$$
\sin x = \sum_{n = 0}^{\infty} \frac{(-1)^n x^{2n+1}}{(2n+1)!}
$$
Разделив обе части равенства на $x$, получим ряд Тейлора для исходной функции:
$$
\frac{\sin x}{x} = \sum_{n = 0}^{\infty} \frac{(-1)^n x^{2n}}{(2n+1)!}
$$
Тогда мы можем выполнить интегрирование исходной функции, вычислив значение интеграла для каждого члена ряда:
$$
\int\limits_{0}^{\frac{1}{2}}\frac{\sin x}{x}dx = \int\limits_{0}^{\frac{1}{2}}dx - \int\limits_{0}^{\frac{1}{2}}\frac{x^2}{3!}dx + \int\limits_{0}^{\frac{1}{2}}\frac{x^4}{5!}dx - ... =
$$
$$
= (x - \frac{x^3}{3\cdot3!} + \frac{x^5}{5\cdot5!} - ...)\bigg|_0^\frac{1}{2} \approx 0.5 - 0.006944 + 0.000052 \approx 0.4931
$$
Для достижения заданной точности хватило сложения трёх членов ряда, так как значение определённого интеграла третьего члена уже меньше 0.0001, а все последующие члены тоже будут меньше этого значения, что не изменит число до заданной точности.
\\
\textbf{Ответ:} 0.4931.

\section*{Задание 3}

Найти в виде степенного ряда решение дифференциального уравнения, удовлетворяющего заданным начальным условиям. Ограничиться четырьмя членами ряда.
$$
y'' = e^{2x}\ln y + 3; y(0) = 1, y'(0) = 2
$$
\textbf{Решение:}
Разложим функцию $y(x)$ в ряд Тейлора в точке $x_0 = 0$:
$$
y = y(0) + \frac{y'(0)x}{1!} + \frac{y''(0)x^2}{2!} + \frac{y'''(0)x^3}{3!}
$$
Мы записали первые четыре члена ряда, поскольку в задании требуется именно столько.
Мы значем значения $y(0)$ и $y'(0)$. Однако нам неизвестны значения $y''(0)$ и $y'''(0)$. Найдём их. Сначала запишем исходное уравнение:
\begin{equation}\label{eqn:1}
    y'' = e^{2x}\ln y + 3\tag{*}
\end{equation}
Подставляя $x = 0$ в \eqref{eqn:1}, получим:
$$
y''(0) = e^0\ln y(0) + 3 = \ln 1 + 3 = 0 + 3 = 3
$$
Дифференцируя обе части \eqref{eqn:1}, получим:
$$
y''' = 2e^{2x}\ln y + e^{2x}\frac{1}{y}y'
$$
Подставим в получившееся уравнение $x = 0$:
$$
y'''(0) = 2e^0\ln y(0) + e^0\frac{1}{y(0)}y'(0) = 0 + 2 = 2
$$
Мы вычислили все небходимые значения производных в точке $x = 0$, поэтому можно подставить их в исходный ряд Тейлора:
$$
y = 1 + 2x + \frac{3}{2}x^2 + \frac{2}{6}x^3 = 1 + 2x + \frac{3}{2}x^2 + \frac{1}{3}x^3
$$
\textbf{Ответ:} $y = 1 + 2x + \frac{3}{2}x^2 + \frac{1}{3}x^3$.
\end{document}
